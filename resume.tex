\documentclass[margin,line]{resume}
\usepackage{lmodern}
\usepackage[hidelinks]{hyperref}

\begin{document}
\name{\Large Mário Feroldi Filho}
\begin{resume}
    \section{\mysidestyle Contact\\Information}

    Phone: +55 18 99796-9367 \hfill \url{github.com/feroldi}\\
    \noindent Email: mferoldif@gmail.com \hfill \url{linkedin.com/in/mferoldif}

    \section{\mysidestyle Education}

    \textbf{Salesiano Auxilium Catholic University}, Araçatuba, Brazil \hfill \textbf{Feb 2015 -- Dec 2019}\\
    \textsl{B.S., Computer Engineering}\\
    \textsl{GPA: 3.45/4.0}

    \section{\mysidestyle Professional\\Experience}

    \textbf{Grupo Salutem}, \textit{Software Engineer Intern} \hfill \textbf{Jan 2019 -- present}
    \begin{itemize}
        \item Optimized the development cycle by moving the compilation, testing and deploying process to the Cloud.
        \item Developed a clinical appointment scheduling mobile application for the AME clinics with the Flutter SDK.
        \item Trained developers to use the Github workflow in the development process.
    \end{itemize}

    \section{\mysidestyle Research\\Experience}

    \textbf{Salesiano Auxilium}, \textsl{Undergraduate Research} \hfill \textbf{Aug 2018 - Jun 2019}
    \begin{itemize}
        \item Researched compiler optimizations in code generation for multiple processor architectures.
        \item Elaborated on instruction selection and the pattern matching and selection problem.
        \item Researched how LLVM, GCC and superoptimizers select instructions and generate quality code.
    \end{itemize}

    \textbf{Salesiano Auxilium}, \textsl{Undergraduate Thesis} \hfill \textbf{Feb 2019 - Dec 2019}
    \begin{itemize}
        \item Researching the Regionalized Value State Dependency Graph (RVSDG) as an intermediate representation (IR) for multiple compiling stages.
        \item Implementing the compiler's back end as a pipeline of transformations on the RVSDG, turning it from higher to lower level at each pass.
        \item Avoiding duplication of effort when implementing common optimizations for different IRs, such as constant propagation.
    \end{itemize}

    \section{\mysidestyle Personal\\Projects}

    \textbf{CCI: C11 Compiler Infrastructure} \hfill \url{github.com/feroldi/cci}
    \begin{itemize}
        \item Developing a compiler infrastructure to manipulate and compile C code.
        \item Meant as a personal research project to acquire hands-on experience on compiler engineering.
        \item Used C++17 for development, GoogleTest for unit tests, and CMake for the build system.
    \end{itemize}

    \textbf{Ruke: A microkernel experiment} \hfill \url{github.com/feroldi/ruke}
    \begin{itemize}
        \item Created a microkernel for the x86 architecture.
        \item An experiment with the Mozilla's Rust programming language.
        \item Used Rust for most of the development, and x86 Assembly for the low-level bits.
    \end{itemize}

    \textbf{Cognita: A flashcard system mobile application} \hfill \url{github.com/feroldi/cognita}
    \begin{itemize}
        \item A system to optimize the learning process of any subject by using the principle of spaced repetition.
        \item Implemented the Leitner system for the training sessions.
        \item Used the Flutter framework, and the SQLite embedded database.
    \end{itemize}

    \section{\mysidestyle Languages}

    Advanced English\\
    Native Portuguese

    \section{\mysidestyle Technical\\Skills}

    Bash, C, C++, C\#, Dart, Flutter, GNU/Linux, Git, Java, POSIX, PostgreSQL, Python, Rust, SQLite, x86 Assembly

    \section{\mysidestyle Others}

    Top 10\% on the C++ tag on Stack Overflow \hfill \textbf{since Jan 2019}\\
    Placed 348th out of 1156 at the XXIII Maratona SBC First Phase \hfill \textbf{Sep 2018}\\
    Undergraduate Teaching Assistant for Compilers \hfill \textbf{Aug 2018}\\
    Gave a presentation to my class on C++14 and its features \hfill \textbf{Aug 2016}\\
    Undergraduate Teaching Assistant for Programming Logic and Algorithms \hfill \textbf{Jan 2015}\\

\end{resume}
\end{document}
